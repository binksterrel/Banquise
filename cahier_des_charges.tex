\documentclass[12pt,a4paper]{report}
\usepackage[utf8]{inputenc}
\usepackage[T1]{fontenc}
\usepackage[french]{babel}
\usepackage{geometry}
\usepackage{enumitem}
\usepackage{hyperref}
\usepackage{graphicx} % gestion des images
\usepackage{float}
\usepackage{tikz}
\usepackage{pgf-umlcd}
\usepackage{booktabs}
\usepackage{longtable}
\usepackage{colortbl}
\usepackage{xcolor}
\usepackage{fancyhdr}
\usepackage{tocloft}
\usepackage{array}
\usepackage{amsmath}  % pour \text, formules...
\usepackage{amssymb}  % pour \checkmark, symboles maths
% Ajout de la bibliothèque pour les diagrammes UML améliorés
\usetikzlibrary{positioning, arrows.meta, shapes, calc}

\geometry{margin=2.5cm}

% === Réglages pour les captures ===
% Vos images seront cherchées dans le dossier ./captures/
\graphicspath{{captures/}}
% Vous pouvez déposer des PNG/JPG/JPEG/PDF sans préciser l'extension
\DeclareGraphicsExtensions{.png,.jpg,.jpeg,.pdf}

% Couleurs personnalisées (charte Banquise)
\definecolor{ice50}{HTML}{F0F9FF}
\definecolor{ice200}{HTML}{BAE6FD}
\definecolor{ice500}{HTML}{0EA5E9}
\definecolor{ink}{HTML}{0F172A}
\definecolor{slate100}{HTML}{E2E8F0}
\definecolor{slate500}{HTML}{64748B}
\definecolor{purple}{HTML}{6366F1}
\definecolor{lightgray}{HTML}{F8FAFC}
\definecolor{banquiseblue}{HTML}{0EA5E9} % compat tables/exemples

% Configuration des liens hypertextes
\hypersetup{
    colorlinks=true,
    linkcolor={ink},
    filecolor={ink},
    urlcolor={ice500},
    citecolor={ice500},
    pdfauthor={Banquise},
    pdftitle={Cahier des Charges - Banquise},
    pdfsubject={Plateforme de néobanque web},
}

% En-têtes et pieds de page
\pagestyle{fancy}
\fancyhf{}
\fancyhead[L]{\leftmark}
\fancyhead[R]{\color{ice500}\textbf{Banquise}}
\fancyfoot[C]{\color{slate500}\thepage}

% Titre
\title{
    \Huge{\textbf{Cahier des Charges}} \\
    \vspace{1cm}
    \LARGE{Banquise} \\
    \vspace{0.5cm}
    \Large{Plateforme de Néobanque Web}
}
\author{Équipe Banquise}
\date{\today}

\begin{document}

% Page de garde personnalisée (charte Banquise)
\begin{titlepage}
\begin{tikzpicture}[remember picture,overlay]
    % Fond dégradé doux bleu / violet
    \fill[ice50] (current page.south west) rectangle (current page.north east);
    \path[fill=ice200] ([xshift=-1cm,yshift=-2cm]current page.north east) circle (7cm);
    \path[fill=purple!20] ([xshift=2cm,yshift=1cm]current page.center) circle (6cm);
    \path[fill=ice500!18] ([xshift=-3cm,yshift=1cm]current page.center) circle (5cm);
    % Bandeau gradient
    \path[fill=ice500!35] ([xshift=-2cm]current page.north west) rectangle ([xshift=2cm,yshift=-2cm]current page.north east);
\end{tikzpicture}

\begin{center}
    \vspace*{2.5cm}
    {\Large\textsc{Banquise}}\\[0.6cm]
    {\huge\bfseries Cahier des Charges}\\[0.4cm]
    {\Large\bfseries Plateforme de Néobanque Web}\\[0.8cm]
    \rule{0.6\linewidth}{0.8pt}\\[0.6cm]
    {\large Version du \today}\\[2.2cm]

    \begin{tabular}{p{3.5cm}p{9cm}}
        \textbf{Produit} : & Banquise (Django 4.2) \\
        \textbf{Équipe} : & Terrel NUENTSA \\[-2pt]
                          & Romain THIERRY \\[-2pt]
                          & Daniel BADOYAN \\
        \textbf{Contact} : & 43020094@parisnanterre.fr \\
    \end{tabular}

    \vfill
    \textit{Banquise : comptes, cartes, virements, crédits avec scoring IA, notifications temps réel, support intégré et console admin.}
\end{center}
\end{titlepage}

\newpage
\thispagestyle{empty}
\tableofcontents

\newpage

% ============================================================================
% CHAPITRE 1 : CONTEXTE GÉNÉRAL DU PROJET
% ============================================================================
\chapter{Contexte général du projet}

\section{Introduction}
Ce document constitue le cahier des charges du projet Banquise, une plateforme de néobanque web développée avec Django 4.2. L'objectif est de fournir une expérience bancaire digitale claire, sécurisée et complète, alignée sur la charte graphique Banquise (dégradés bleu/ice, typographie épurée) et intégrant les flux suivants :
\begin{itemize}[leftmargin=1.3cm]
    \item Comptes et cartes (blocage, plafonds, RIB/IBAN, relevés PDF)
    \item Virements et bénéficiaires (ajout, modification, suppression sécurisée)
    \item Abonnements (Essentiel / Plus / Infinite) avec confirmation par mot de passe et résiliation encadrée
    \item Crédits avec scoring IA, simulations, validations admin et versement automatique sur compte
    \item Notifications temps réel (pastille dock) et toasts temporisés
    \item Support (chat, FAQ, avis utilisateur) et console admin (gestion crédits, comptes, cartes)
\end{itemize}

Le projet s'inscrit dans une démarche de transformation numérique du secteur bancaire, en proposant une solution moderne, accessible et centrée sur l'utilisateur.

\section{Présentation de l'organisme}
\subsection{Présentation générale}
Banquise est une néobanque française visant à démocratiser l'accès aux services bancaires par le biais d'une plateforme web intuitive et sécurisée. Le projet est actuellement en mode démo avec une trajectoire d'industrialisation progressive.

\subsection{Vision et mission}
\begin{itemize}[leftmargin=1.5cm]
    \item \textbf{Vision} : Devenir la néobanque de référence pour les particuliers en France
    \item \textbf{Mission} : Offrir des services bancaires accessibles, transparents et innovants
    \item \textbf{Valeurs} : Simplicité, sécurité, transparence, innovation
\end{itemize}

\section{Présentation du projet}

\subsection{Analyse de l'existant}
Actuellement, le secteur bancaire traditionnel présente plusieurs limitations :
\begin{itemize}[leftmargin=1.5cm]
    \item Processus lourds et bureaucratiques
    \item Frais bancaires élevés et peu transparents
    \item Interface utilisateur complexe et peu intuitive
    \item Délais de traitement importants pour les opérations
    \item Faible réactivité du service client
\end{itemize}

\subsection{Problématique}
\textbf{Comment concevoir et mettre en œuvre une plateforme bancaire digitale moderne qui offre une expérience utilisateur optimale, une sécurité renforcée et des services innovants tout en garantissant la conformité réglementaire et la scalabilité du système ?}

\subsection{Solution proposée}
Banquise propose une plateforme web complète comprenant :
\begin{itemize}[leftmargin=1.5cm]
    \item Gestion multi-comptes (courant, épargne) et cartes virtuelles avec contrôle (blocage, plafonds)
    \item Virements instantanés, ajout/modification/suppression de bénéficiaires avec validations
    \item Abonnements flexibles (Essentiel, Plus, Infinite) avec confirmation par mot de passe
    \item Simulation et demande de crédit avec scoring IA, historique et versement automatique en cas d'acceptation
    \item Prévision des mensualités crédit dans les analyses de dépenses (courbes superposées)
    \item Notifications en temps réel (dock et pastille) et toasts à durée limitée
    \item Support client (chat, FAQ enrichie, avis affichés sur la page d'accueil)
    \item Console administrateur : gestion des crédits (acceptation, annulation, modification des paramètres), comptes, cartes et bénéficiaires
\end{itemize}

\subsection{Alignement UX / UI (site actuel)}
\begin{itemize}[leftmargin=1.5cm]
    \item \textbf{Charte} : palette Ice / Indigo, fonds dégradés, cartes glassmorphism, pastille de notifications.
    \item \textbf{Navigation} : dock flottant (masqué sur login/register), barre de progression de scroll, navbar sticky.
    \item \textbf{Sécurité UX} : formulaires protégés (pas d'actions sans connexion), toasts temporisés, anti-cache sur pages protégées.
    \item \textbf{Support} : FAQ + filtres (Opposition carte, Sécurité, Virements, Attestations, IA \& scoring) et chat support.
    \item \textbf{Crédits} : historique verrouillé après acceptation, projections de mensualités dans les graphiques dépenses, console admin dédiée (/console/credits/).
\end{itemize}

\subsection{Choix de la méthodologie}
Le projet adopte une \textbf{méthodologie Agile} avec des itérations courtes permettant :
\begin{itemize}[leftmargin=1.5cm]
    \item Livraisons incrémentales de fonctionnalités
    \item Adaptation rapide aux retours utilisateurs
    \item Amélioration continue de la qualité
    \item Gestion flexible des priorités
\end{itemize}

\section{Conclusion}
Ce chapitre a posé les fondations du projet Banquise en présentant le contexte, la problématique et la solution proposée. Les chapitres suivants détailleront les besoins fonctionnels, l'architecture technique et les spécifications détaillées.

% ============================================================================
% CHAPITRE 2 BIS : IA ET SCORING CRÉDIT BANQUISE
% ============================================================================
\chapter{IA et scoring crédit Banquise}

\section{Objectifs}
L'IA Banquise vise à :
\begin{itemize}[leftmargin=1.3cm]
    \item Fournir une pré-décision immédiate (avis automatique) lors des simulations de crédit.
    \item Calculer des indicateurs clés : taux d'endettement (DTI), ratio prêt/valeur (LTV), marge de sécurité.
    \item Proposer une recommandation ajustée (durée, mensualité cible) si les seuils ne sont pas respectés.
    \item Alimenter la console admin et les notifications pour accélérer la validation manuelle.
\end{itemize}

\section{Sources de données}
\begin{itemize}[leftmargin=1.3cm]
    \item Données saisies par l'utilisateur : revenus, charges, situation professionnelle, apport, durée souhaitée.
    \item Historique client Banquise : comptes actifs, solde, transactions, crédits existants, abonnements.
    \item Paramètres produit : taux de référence, grilles de scoring internes, bonus (ex. propriétaire).
\end{itemize}

\section{Calculs principaux}
\begin{itemize}[leftmargin=1.3cm]
    \item \textbf{Mensualité théorique} : formule d'annuité (taux mensuel, durée, montant).
    \item \textbf{DTI} (Debt-to-Income) : (mensualité + charges) / revenus nets.
    \item \textbf{LTV} (Loan-to-Value) : montant prêté / valeur du bien (ou apport total).
    \item \textbf{Score IA} : pondération des critères (DTI, LTV, stabilité pro, bonus propriétaire, historique Banquise).
    \item \textbf{Recommandation} : si dépassement de seuils, proposition d'une durée ajustée et d'une mensualité cible.
\end{itemize}

\section{Seuils et décisions}
\begin{itemize}[leftmargin=1.3cm]
    \item DTI seuil : 40\% (ajustable).
    \item LTV seuil : 90\% (ajustable).
    \item Score minimal : 70/100 pour une acceptation automatique.
    \item Cas limites : si DTI légèrement au-dessus, l'IA suggère d'allonger la durée ou d'augmenter l'apport.
\end{itemize}

\section{Parcours utilisateur}
\begin{enumerate}[leftmargin=1.3cm]
    \item L'utilisateur remplit la simulation (montant, durée, revenus, charges, apport).
    \item L'IA calcule DTI, LTV, mensualité, score et rend un \textit{avis automatique} (accepté / ajusté / refusé).
    \item Si l'utilisateur valide, la demande passe en \textbf{EN\_ATTENTE} pour revue admin.
    \item En cas d'acceptation admin, le montant est crédité sur le compte et les mensualités sont planifiées (débits mensuels).
    \item L'historique et l'analyse des dépenses affichent la projection des mensualités jusqu'à échéance.
\end{enumerate}

\section{Intégration produit}
\begin{itemize}[leftmargin=1.3cm]
    \item \textbf{Dashboard} : courbe des dépenses incluant les mensualités crédit (multi-couleurs si plusieurs crédits).
    \item \textbf{Simulation} : avis IA affiché, avec suggestion de durée/mensualité quand nécessaire.
    \item \textbf{Historique} : statut verrouillé si crédit accepté ; modification impossible côté client.
    \item \textbf{Console admin} : modification des paramètres du crédit (montant/durée) et annulation possible même après acceptation.
\end{itemize}

\section{Sécurité et conformité}
\begin{itemize}[leftmargin=1.3cm]
    \item Traçabilité des décisions IA (score, critères, horodatage) pour audit interne.
    \item Notifications aux utilisateurs et aux admins lors des changements de statut.
    \item Aucune donnée externe sensible transmise : calculs réalisés côté serveur.
\end{itemize}

\section{Limites et évolutions}
\begin{itemize}[leftmargin=1.3cm]
    \item Version actuelle : moteur heuristique + règles métiers (DTI/LTV/score pondéré).
    \item Évolution envisagée : modèle ML supervisé avec jeu de données anonymisé, calibration des seuils, détection de risques.
    \item Explicabilité : conserver les raisons d'un refus (DTI/LTV/score) pour transparence utilisateur.
\end{itemize}

% ============================================================================
% CHAPITRE 2 : ANALYSE ET SPÉCIFICATION DES BESOINS
% ============================================================================
\chapter{Analyse et spécification des besoins}

\section{Introduction}
Ce chapitre présente l'analyse détaillée des besoins fonctionnels et non fonctionnels du système, ainsi que l'identification des acteurs et leurs rôles.

\section{Identification des acteurs}

Le système Banquise implique trois acteurs principaux :

\begin{table}[H]
\centering
\begin{tabular}{|p{3cm}|p{11cm}|}
\hline
\rowcolor{banquiseblue!30}
\textbf{Acteur} & \textbf{Rôle et responsabilités} \\
\hline
\textbf{Client} & 
\begin{itemize}[leftmargin=0.5cm]
    \item Ouvre et gère ses comptes bancaires
    \item Gère ses cartes (blocage, plafonds)
    \item Effectue des virements
    \item Simule et demande des crédits
    \item Consulte ses notifications
    \item Contacte le support
\end{itemize} \\
\hline
\textbf{Staff (Admin)} & 
\begin{itemize}[leftmargin=0.5cm]
    \item Supervise l'activité globale
    \item Valide ou refuse les demandes de crédit
    \item Gère les comptes, cartes et transactions
    \item Répond aux demandes de support
    \item Accède au tableau de bord administratif
\end{itemize} \\
\hline
\textbf{Système} & 
\begin{itemize}[leftmargin=0.5cm]
    \item Calcule l'avis automatique de crédit
    \item Applique les règles de découvert
    \item Génère les notifications
    \item Sécurise les transactions
    \item Gère les abonnements automatiques
\end{itemize} \\
\hline
\end{tabular}
\caption{Identification des acteurs du système}
\end{table}

\section{Glossaire}

\begin{description}[style=nextline, leftmargin=3cm]
    \item[Découvert autorisé] Plafond de solde négatif toléré selon l'abonnement (Essentiel : 100€, Plus : 500€, Infinite : 1000€)
    \item[DTI] Debt-to-Income : ratio d'endettement mensuel calculé pour l'évaluation de crédit
    \item[LTV] Loan-to-Value : ratio entre le montant du prêt et la valeur du bien financé
    \item[Scoring] Algorithme d'évaluation automatique de la solvabilité d'un client
    \item[Notification] Message système stocké en base avec statut lu/non lu
    \item[IBAN] International Bank Account Number : numéro de compte bancaire international
    \item[BIC] Bank Identifier Code : code d'identification de la banque
\end{description}

\section{Besoins fonctionnels}

\subsection{Module Authentification}
\begin{itemize}[leftmargin=1.5cm]
    \item Inscription avec validation des informations
    \item Connexion sécurisée
    \item Déconnexion
    \item Gestion du profil utilisateur
    \item Changement de mot de passe
    \item Réinitialisation de mot de passe
\end{itemize}

\subsection{Module Gestion des Comptes}
\begin{itemize}[leftmargin=1.5cm]
    \item Ouverture de compte (Courant, Épargne, Pro)
    \item Fermeture de compte
    \item Consultation du solde en temps réel
    \item Relevé de compte paginé
    \item Export PDF du relevé
    \item Génération de RIB
    \item Consultation de l'IBAN et BIC
\end{itemize}

\subsection{Module Cartes Bancaires}
\begin{itemize}[leftmargin=1.5cm]
    \item Affichage des cartes associées
    \item Blocage/déblocage manuel des cartes
    \item Activation/désactivation sans-contact
    \item Activation/désactivation paiements étrangers
    \item Gestion des plafonds de paiement
    \item Gestion des plafonds de retrait
    \item Blocage automatique en cas de dépassement du découvert autorisé
    \item Déblocage automatique lorsque le solde repasse au-dessus
\end{itemize}

\subsection{Module Virements}
\begin{itemize}[leftmargin=1.5cm]
    \item Ajout de bénéficiaires (IBAN)
    \item Gestion des bénéficiaires enregistrés
    \item Virement vers bénéficiaire existant
    \item Virement vers IBAN saisi manuellement
    \item Vérification du solde disponible
    \item Transaction miroir pour virements internes Banquise
    \item Normalisation IBAN (insensible casse et espaces)
    \item Notifications émetteur et destinataire
\end{itemize}

\subsection{Module Abonnements}
\begin{itemize}[leftmargin=1.5cm]
    \item Plans disponibles : Essentiel (gratuit), Plus (9,90€), Infinite (19,90€)
    \item Changement d'abonnement avec débit immédiat
    \item Création d'une transaction associée
    \item Prochaine facturation à J+30
    \item Résiliation programmée en fin de période
    \item Notifications de changement d'abonnement
    \item Découvert autorisé selon le plan
\end{itemize}

\subsection{Module Crédit}
\begin{itemize}[leftmargin=1.5cm]
    \item Simulation de crédit avec formulaire détaillé
    \item Calcul automatique du score de solvabilité
    \item Avis IA basé sur DTI, LTV et tolérance au risque
    \item Statut initial « En attente de validation humaine »
    \item Historique des demandes de crédit
    \item Page résultat dédiée avec détails
    \item Validation/refus par administrateur
    \item Notifications à chaque étape
\end{itemize}

\subsection{Module Notifications}
\begin{itemize}[leftmargin=1.5cm]
    \item Centre de notifications utilisateur
    \item Badge sur l'avatar indiquant les notifications non lues
    \item Notifications pour virements reçus/émis
    \item Notifications pour transactions d'abonnement
    \item Alertes de dépassement de découvert
    \item Notifications de décision de crédit
    \item Notifications de messages support
    \item Marquage en lu
\end{itemize}

\subsection{Module Support Client}
\begin{itemize}[leftmargin=1.5cm]
    \item Chat bidirectionnel client $\leftrightarrow$ admin
    \item Envoi de messages par le client
    \item Réponses des administrateurs
    \item Notifications côté client et staff
    \item Liste des conversations pour les admins
    \item Filtrage par utilisateur
\end{itemize}

\subsection{Module Administration}
\begin{itemize}[leftmargin=1.5cm]
    \item Dashboard avec statistiques clés :
    \begin{itemize}
        \item Nombre d'utilisateurs
        \item Comptes actifs
        \item Solde total de la plateforme
        \item Crédits en attente
        \item Transactions récentes
        \item Top dépenses
    \end{itemize}
    \item Console de validation des crédits
    \item Gestion des comptes/cartes/bénéficiaires
    \item Consultation des transactions
    \item Accès Django admin pour CRUD complet
\end{itemize}

\section{Besoins non fonctionnels}

\begin{table}[H]
\centering
\begin{tabular}{|p{4cm}|p{10cm}|}
\hline
\rowcolor{banquiseblue!30}
\textbf{Critère} & \textbf{Exigences} \\
\hline
\textbf{Sécurité} & 
\begin{itemize}[leftmargin=0.5cm]
    \item Authentification Django sécurisée
    \item Protection CSRF activée
    \item Gestion des rôles (staff/client)
    \item HTTPS recommandé en production
    \item Rotation des clés secrètes
    \item 2FA pour les administrateurs
    \item Journal d'audit des actions sensibles
\end{itemize} \\
\hline
\textbf{Performance} & 
\begin{itemize}[leftmargin=0.5cm]
    \item Pagination sur les relevés (20 items/page)
    \item Limitation des listes (30 éléments max)
    \item Calculs de simulation côté serveur
    \item Temps de réponse < 2 secondes
\end{itemize} \\
\hline
\textbf{Disponibilité} & 
\begin{itemize}[leftmargin=0.5cm]
    \item Journalisation des événements clés
    \item Sauvegardes régulières de la base
    \item Alertes sur erreurs critiques
    \item Disponibilité cible : 99\%
\end{itemize} \\
\hline
\textbf{Scalabilité} & 
\begin{itemize}[leftmargin=0.5cm]
    \item Base PostgreSQL en production
    \item Stockage statique dédié (CDN)
    \item Files d'attente pour notifications
    \item Architecture horizontalement scalable
\end{itemize} \\
\hline
\textbf{Conformité} & 
\begin{itemize}[leftmargin=0.5cm]
    \item Respect du RGPD
    \item Données personnelles minimales
    \item Droits d'accès et de suppression
    \item Traçabilité des décisions de crédit
\end{itemize} \\
\hline
\textbf{Ergonomie} & 
\begin{itemize}[leftmargin=0.5cm]
    \item Interface responsive
    \item Design moderne avec Tailwind CSS
    \item Navigation intuitive
    \item Accessibilité (WCAG 2.1)
\end{itemize} \\
\hline
\end{tabular}
\caption{Besoins non fonctionnels}
\end{table}

\section{Diagramme de cas d'utilisation général}
\begin{figure}[H]
\centering
\begin{tikzpicture}[
    scale=0.7, transform shape,
    actor/.style={draw, shape=circle, minimum size=0.9cm, font=\small, fill=banquiseblue!20},
    usecase/.style={draw, ellipse, minimum width=2.8cm, minimum height=1.1cm, font=\scriptsize, fill=lightgray},
    sys/.style={thick, rounded corners=8pt, fill=white}
]
    \node[actor] (client) at (-2,0) {Client};
    \node[actor] (admin) at (-2,-7) {Admin};

    \node[sys, minimum width=12cm, minimum height=11cm, anchor=center] (box) at (5,-4) {};
    \node[font=\small, anchor=north] at (5,1.3) {\textbf{Système Banquise}};

    \node[usecase] (auth) at (3,0.5) {S'authentifier};
    \node[usecase] (compte) at (7,0.5) {Gérer comptes};
    \node[usecase] (carte) at (3,-1.4) {Gérer cartes};
    \node[usecase] (virement) at (7,-1.4) {Faire virement};
    \node[usecase] (credit) at (3,-3.3) {Simuler \& demander crédit};
    \node[usecase] (notif) at (7,-3.3) {Voir notifications};
    \node[usecase] (support) at (5,-5.1) {Contacter support};

    \node[usecase, fill=banquiseblue!15] (dash) at (3,-6.8) {Dashboard admin};
    \node[usecase, fill=banquiseblue!15] (valid) at (7,-6.8) {Valider crédits};
    \node[usecase, fill=banquiseblue!15] (manage) at (5,-8.3) {Gérer système \& support};

    \draw[-] (client) -- (auth);
    \draw[-] (client) -- (compte);
    \draw[-] (client) -- (carte);
    \draw[-] (client) -- (virement);
    \draw[-] (client) -- (credit);
    \draw[-] (client) -- (notif);
    \draw[-] (client) -- (support);

    \draw[-] (admin) -- (dash);
    \draw[-] (admin) -- (valid);
    \draw[-] (admin) -- (manage);
    \draw[-] (admin) -- (support);
\end{tikzpicture}
\caption{Diagramme de cas d'utilisation général (Figure 2.1)}
\end{figure}

\section{Arborescence du site}
\begin{figure}[H]
\centering
\begin{tikzpicture}[
  scale=0.8,
  transform shape,
  level 1/.style={sibling distance=4cm, level distance=1.8cm},
  level 2/.style={sibling distance=2.2cm, level distance=1.4cm},
  edge from parent/.style={draw, -latex, thick},
  every node/.style={font=\scriptsize, inner sep=3pt}
]

\node[draw, rectangle, fill=banquiseblue!30, minimum width=1.8cm, minimum height=0.7cm] {Accueil}
  child {node[draw, rectangle, fill=lightgray, minimum width=1.6cm, minimum height=0.6cm] {Espace Client}
    child {node[draw, rectangle, minimum width=1.6cm, minimum height=0.5cm] {Dashboard}}
    child {node[draw, rectangle, minimum width=1.6cm, minimum height=0.5cm] {Comptes}}
    child {node[draw, rectangle, minimum width=1.6cm, minimum height=0.5cm] {Cartes}}
    child {node[draw, rectangle, minimum width=1.6cm, minimum height=0.5cm] {Notifications}}
  }
  child {node[draw, rectangle, fill=lightgray, minimum width=1.6cm, minimum height=0.6cm] {Virements}
    child {node[draw, rectangle, minimum width=1.6cm, minimum height=0.5cm] {Bénéficiaires}}
    child {node[draw, rectangle, minimum width=1.6cm, minimum height=0.5cm] {Nouveau virement}}
  }
  child {node[draw, rectangle, fill=lightgray, minimum width=1.6cm, minimum height=0.6cm] {Services}
    child {node[draw, rectangle, minimum width=1.6cm, minimum height=0.5cm] {Crédit (simulation)}}
    child {node[draw, rectangle, minimum width=1.6cm, minimum height=0.5cm] {Abonnements}}
    child {node[draw, rectangle, minimum width=1.6cm, minimum height=0.5cm] {Support \& chat}}
  }
  child {node[draw, rectangle, fill=banquiseblue!15, minimum width=1.6cm, minimum height=0.6cm] {Admin}
    child {node[draw, rectangle, minimum width=1.6cm, minimum height=0.5cm] {Dashboard admin}}
    child {node[draw, rectangle, minimum width=1.6cm, minimum height=0.5cm] {Crédits (validation)}}
    child {node[draw, rectangle, minimum width=1.6cm, minimum height=0.5cm] {Console gestion}}
    child {node[draw, rectangle, minimum width=1.6cm, minimum height=0.5cm] {Rapports/Heatmap}}
  };
\end{tikzpicture}
\caption{Arborescence du site (Figure 2.2)}
\end{figure}

\section{User Stories principales}

\begin{longtable}{|c|p{2.5cm}|p{7cm}|c|c|}
\hline
\rowcolor{banquiseblue!30}
\textbf{N°} & \textbf{Module} & \textbf{User Story} & \textbf{Priorité} & \textbf{Points} \\
\hline
\endfirsthead
\hline
\rowcolor{banquiseblue!30}
\textbf{N°} & \textbf{Module} & \textbf{User Story} & \textbf{Priorité} & \textbf{Points} \\
\hline
\endhead

1 & Authentification & En tant qu'utilisateur, je veux m'inscrire avec mes informations personnelles & Haute & 8 \\
\hline
2 & Authentification & En tant qu'utilisateur, je veux me connecter de manière sécurisée & Haute & 5 \\
\hline
3 & Authentification & En tant qu'utilisateur, je veux changer mon mot de passe & Haute & 3 \\
\hline
4 & Comptes & En tant que client, je veux ouvrir un nouveau compte & Haute & 5 \\
\hline
5 & Comptes & En tant que client, je veux consulter mes soldes en temps réel & Haute & 3 \\
\hline
6 & Comptes & En tant que client, je veux consulter mon relevé de compte & Moyenne & 5 \\
\hline
7 & Comptes & En tant que client, je veux exporter mon relevé en PDF & Basse & 8 \\
\hline
8 & Cartes & En tant que client, je veux bloquer/débloquer ma carte & Haute & 3 \\
\hline
9 & Cartes & En tant que client, je veux gérer mes plafonds de paiement & Moyenne & 5 \\
\hline
10 & Cartes & En tant que système, je veux bloquer automatiquement les cartes en cas de dépassement & Haute & 8 \\
\hline
11 & Virements & En tant que client, je veux ajouter un bénéficiaire & Haute & 3 \\
\hline
12 & Virements & En tant que client, je veux effectuer un virement & Haute & 5 \\
\hline
13 & Virements & En tant que client, je veux être notifié des virements reçus & Haute & 3 \\
\hline
14 & Abonnements & En tant que client, je veux changer mon abonnement & Moyenne & 5 \\
\hline
15 & Abonnements & En tant que client, je veux résilier mon abonnement & Moyenne & 3 \\
\hline
16 & Crédit & En tant que client, je veux simuler un crédit & Haute & 13 \\
\hline
17 & Crédit & En tant que système, je veux calculer un score de solvabilité & Haute & 13 \\
\hline
18 & Crédit & En tant qu'admin, je veux valider ou refuser un crédit & Haute & 5 \\
\hline
19 & Notifications & En tant que client, je veux consulter mes notifications & Haute & 3 \\
\hline
20 & Notifications & En tant que client, je veux voir un badge de notifications non lues & Moyenne & 2 \\
\hline
21 & Support & En tant que client, je veux envoyer un message au support & Moyenne & 3 \\
\hline
22 & Support & En tant qu'admin, je veux répondre aux messages clients & Moyenne & 5 \\
\hline
23 & Admin & En tant qu'admin, je veux consulter le dashboard avec statistiques & Haute & 8 \\
\hline
24 & Admin & En tant qu'admin, je veux gérer les comptes/cartes & Haute & 8 \\
\hline

\caption{Backlog des User Stories}
\end{longtable}

% ============================================================================
% CHAPITRE 3 : ARCHITECTURE TECHNIQUE ET SÉCURITÉ
% ============================================================================
\chapter{Architecture technique et sécurité}

\section{Architecture cible}
\begin{itemize}[leftmargin=1.5cm]
    \item \textbf{Backend} : Django 4.2, Python 3.9+
    \item \textbf{Base de données} : SQLite en développement, PostgreSQL recommandé en production
    \item \textbf{Front} : Templates Django + Tailwind CDN + Bootstrap Icons
    \item \textbf{Notifications} : modèle interne (table Notification), extensible vers e-mail/SMS
    \item \textbf{Stockage statique} : collectstatic vers un bucket/CDN en production
    \item \textbf{Tâches planifiées} : commande cron pour rapport hebdo (crédits, comptes à risque)
    \item \textbf{Monitoring} : journaux applicatifs (errors, accès), alertes sur erreurs critiques
\end{itemize}

\section{Sécurité}
\begin{itemize}[leftmargin=1.5cm]
    \item Authentification Django sécurisée, CSRF activé, rôles (client/staff)
    \item Découvert : blocage/déblocage automatique des cartes, notifications préventives
    \item Scoring crédit : décision IA conservée, validation humaine obligatoire
    \item Cookies sécurisés en production (HTTPS, Secure, SameSite), rotation de la SECRET\_KEY
    \item 2FA recommandé pour les comptes staff et accès back-office
    \item Journal d'audit des actions sensibles (blocage carte, validation crédit, changement d'abonnement)
    \item Conformité RGPD : minimisation des données, droit de suppression, consentement explicite pour notifications
\end{itemize}

\section{Règles métiers clés}
\begin{itemize}[leftmargin=1.5cm]
    \item Découvert autorisé : Essentiel 100 €, Plus 500 €, Infinite 1000 € ; relèvement temporaire sur demande admin
    \item Scoring crédit : DTI, LTV, score IA, seuils dynamiques selon revenus et montants
    \item Virements : miroir interne si IBAN Banquise, normalisation IBAN (espaces/casse ignorés), notification émetteur/destinataire
    \item Abonnements : débit immédiat, prochaine facturation J+30, résiliation fin de période
\end{itemize}

% ============================================================================
% CHAPITRE 4 : PLANIFICATION ET LIVRABLES
% ============================================================================
\chapter{Planification et livrables}

\section{Jalons}
\begin{itemize}[leftmargin=1.5cm]
    \item \textbf{MVP} : Auth, comptes, virements, cartes (blocage manuel), notifications de base
    \item \textbf{V1} : Abonnements, scoring crédit IA, validation admin, support chat
    \item \textbf{V2} : Découvert dynamique, rapports admin (heatmap, comptes à risque), tâches planifiées
    \item \textbf{V3} : Déploiement cloud (PostgreSQL, stockage statique), durcissement sécurité (2FA, HSTS)
\end{itemize}

\section{Livrables}
\begin{itemize}[leftmargin=1.5cm]
    \item Code source Django + templates
    \item Base de données migrée (migrations Django)
    \item Documentation d’installation (dev/prod) et d’exploitation (collectstatic, migrations, tâches cron)
    \item Cahier des charges (ce document) et guides utilisateur/admin
    \item Jeux de tests (unitaires/integ) et rapports de test
\end{itemize}

% ============================================================================
% CHAPITRE 5 : QUALITÉ ET TESTS
% ============================================================================
\chapter{Qualité et tests}

\section{Types de tests}
\begin{itemize}[leftmargin=1.5cm]
    \item \textbf{Unitaires} : calculs métier (scoring, découvert, miroir virement)
    \item \textbf{Intégration} : parcours clés (login, virement interne/externe, changement d’abonnement, blocage carte)
    \item \textbf{End-to-end} : navigation dashboard, simulation crédit jusqu’à validation admin
    \item \textbf{Performance} : temps de réponse des pages critiques < 2s (dashboard, virement)
    \item \textbf{Sécurité} : CSRF, sessions, droits staff vs client, audit des actions sensibles
\end{itemize}

\section{Critères d’acceptation principaux}
\begin{itemize}[leftmargin=1.5cm]
    \item Ouverture de compte, virement, blocage carte et changement d’abonnement fonctionnent sans erreur
    \item Le scoring IA produit un score et une décision, mais l’admin reste décisionnaire
    \item Les notifications critiques (virement, crédit, découvert) sont créées et visibles
    \item Les tâches planifiées (rapport hebdo) s’exécutent sans erreur
\end{itemize}

\section{Règles métier}

\subsection{Découverts autorisés}
\begin{table}[H]
\centering
\begin{tabular}{|l|c|c|}
\hline
\rowcolor{banquiseblue!30}
\textbf{Abonnement} & \textbf{Prix mensuel} & \textbf{Découvert autorisé} \\
\hline
Essentiel & 0,00 € & 100 € \\
\hline
Plus & 9,90 € & 500 € \\
\hline
Infinite & 19,90 € & 1 000 € \\
\hline
\end{tabular}
\caption{Découverts autorisés par abonnement}
\end{table}

\subsection{Blocage automatique des cartes}
\begin{itemize}[leftmargin=1.5cm]
    \item Si solde $<$ $-$limite découvert : blocage automatique de toutes les cartes
    \item Si solde $\geq$ $-$limite découvert : déblocage automatique
    \item Notification envoyée à chaque changement de statut
    \item Alerte préventive à 80\% de la limite
\end{itemize}

\subsection{Virements internes}
\begin{itemize}[leftmargin=1.5cm]
    \item Détection automatique des IBAN Banquise (FR76)
    \item Création d'une transaction miroir en crédit pour le destinataire
    \item Notification émetteur : « Virement de X€ effectué »
    \item Notification destinataire : « Virement de X€ reçu »
    \item Normalisation IBAN : insensible à la casse et aux espaces
\end{itemize}

\subsection{Scoring crédit}
Le calcul du score de crédit prend en compte :
\begin{itemize}[leftmargin=1.5cm]
    \item \textbf{DTI (Debt-to-Income)} : ratio dettes/revenus
    \item \textbf{LTV (Loan-to-Value)} : ratio montant prêt/valeur financée
    \item \textbf{Situation professionnelle} : CDI, CDD, indépendant, retraité
    \item \textbf{État de santé} : bon, moyen, faible
    \item \textbf{Nombre d'enfants à charge}
    \item \textbf{Apport personnel}
\end{itemize}

\textbf{Formule de scoring} :
\[
\text{Score} = 1000 - (\text{DTI} \times 10) - (\text{LTV} \times 5) + \text{bonus\_emploi} + \text{bonus\_santé}
\]

\subsection{Cycle de vie du crédit}
\begin{enumerate}
    \item Client remplit le formulaire de simulation
    \item Système calcule le score et génère un avis IA
    \item Statut initial : « EN\_ATTENTE »
    \item Notification envoyée au client et aux admins
    \item Admin examine la demande et décide
    \item Statut mis à jour : « ACCEPTEE » ou « REFUSEE »
    \item Notification de décision envoyée au client
\end{enumerate}

\section{Conclusion}
Ce chapitre a présenté l'ensemble des besoins fonctionnels et non fonctionnels, les acteurs, les user stories et les règles métier. Le chapitre suivant détaillera l'architecture technique et la conception du système.

% ============================================================================
% CHAPITRE 3 : ARCHITECTURE ET CONCEPTION
% ============================================================================
\chapter{Architecture et conception}

\section{Introduction}
Ce chapitre présente l'architecture logique et physique du système, ainsi que les diagrammes UML de conception.

\section{Architecture logique}

\subsection{Architecture MVC Django}

% Réduction de la taille et réorganisation pour meilleur affichage
\begin{figure}[H]
\centering
\begin{tikzpicture}[node distance=2.2cm, auto, >=latex', font=\small]
    % Définition des styles
    \tikzstyle{block} = [rectangle, draw, fill=blue!20, text width=4.5em, text centered, rounded corners, minimum height=2.5em]
    \tikzstyle{line} = [draw, -latex', thick]
    
    % Nœuds
    \node [block] (client) {Client Web};
    \node [block, below of=client, node distance=2cm] (urls) {URLs Router};
    \node [block, below of=urls, node distance=2cm] (views) {Views};
    \node [block, left of=views, node distance=3.5cm] (models) {Models};
    \node [block, right of=views, node distance=3.5cm] (templates) {Templates};
    \node [block, below of=views, node distance=2cm] (db) {Base de données};
    
    % Connexions
    \path [line] (client) -- node[right, font=\scriptsize] {HTTP Request} (urls);
    \path [line] (urls) -- (views);
    \path [line] (views) -- (models);
    \path [line] (models) -- (db);
    \path [line] (views) -- (templates);
    \path [line] (templates) -- node[right, font=\scriptsize] {HTTP Response} (client);
    \path [line] (db) -- (models);
\end{tikzpicture}
\caption{Architecture MVC Django}
\end{figure}

\subsection{Structure des modules}

\begin{table}[H]
\centering
\begin{tabular}{|p{4cm}|p{10cm}|}
\hline
\rowcolor{banquiseblue!30}
\textbf{Module} & \textbf{Responsabilités} \\
\hline
\texttt{scoring.models} & Définition des modèles de données (User, Compte, Carte, Transaction, etc.) \\
\hline
\texttt{scoring.views} & Logique métier et contrôleurs (dashboard, virements, crédit, etc.) \\
\hline
\texttt{scoring.forms} & Formulaires Django (inscription, virement, simulation crédit) \\
\hline
\texttt{scoring.urls} & Configuration des routes URL \\
\hline
\texttt{scoring.utils} & Fonctions utilitaires (calcul découvert, scoring, normalisation IBAN) \\
\hline
\texttt{templates/} & Templates HTML avec Tailwind CSS et Bootstrap Icons \\
\hline
\texttt{static/} & Fichiers CSS, JavaScript et images statiques \\
\hline
\end{tabular}
\caption{Structure des modules}
\end{table}

\section{Architecture physique}

\subsection{Architecture client-serveur}

% Simplification du diagramme client-serveur
\begin{figure}[H]
\centering
\begin{tikzpicture}[node distance=2.5cm, auto, font=\small]
    \tikzstyle{block} = [rectangle, draw, fill=green!20, text width=5em, text centered, minimum height=3em]
    \tikzstyle{line} = [draw, -latex', thick]
    
    % Clients
    \node [block] (browser) {Navigateur Web};
    \node [block, below of=browser, node distance=1.8cm] (mobile) {Mobile};
    
    % Serveur Web
    \node [block, right of=browser, node distance=4.5cm, yshift=-0.9cm] (nginx) {Nginx};
    
    % Serveur Application
    \node [block, right of=nginx, node distance=4.5cm] (django) {Django + Gunicorn};
    
    % Base de données
    \node [block, below of=django, node distance=2.5cm] (postgres) {PostgreSQL};
    
    % Stockage
    \node [block, above of=django, node distance=2.5cm] (storage) {Stockage};
    
    % Connexions
    \path [line] (browser) -- node[above, font=\scriptsize] {HTTPS} (nginx);
    \path [line] (mobile) -- (nginx);
    \path [line] (nginx) -- node[above, font=\scriptsize] {WSGI} (django);
    \path [line] (django) -- node[right, font=\scriptsize] {SQL} (postgres);
    \path [line] (django) -- (storage);
\end{tikzpicture}
\caption{Architecture client-serveur}
\end{figure}

\subsection{Diagramme de déploiement}

% Réduction de la taille du diagramme de déploiement
\begin{figure}[H]
\centering
\begin{tikzpicture}[node distance=2cm, font=\small]
    % Serveur Web
    \node[draw, rectangle, minimum width=3.5cm, minimum height=2.5cm] (web) at (0,0) {};
    \node at (0,1.1) {\textbf{Serveur Web}};
    \node[draw, rectangle, fill=yellow!20, font=\scriptsize] at (0,0.3) {Nginx};
    \node[draw, rectangle, fill=yellow!20, font=\scriptsize] at (0,-0.4) {SSL};
    
    % Serveur Application
    \node[draw, rectangle, minimum width=3.5cm, minimum height=2.5cm] (app) at (5,0) {};
    \node at (5,1.1) {\textbf{Serveur App}};
    \node[draw, rectangle, fill=orange!20, font=\scriptsize] at (5,0.3) {Django 4.2};
    \node[draw, rectangle, fill=orange!20, font=\scriptsize] at (5,-0.4) {Gunicorn};
    
    % Base de données
    \node[draw, rectangle, minimum width=3.5cm, minimum height=2.5cm] (db) at (5,-3.5) {};
    \node at (5,-2.4) {\textbf{Base de données}};
    \node[draw, rectangle, fill=blue!20, font=\scriptsize] at (5,-3.2) {PostgreSQL};
    \node[draw, rectangle, fill=blue!20, font=\scriptsize] at (5,-3.9) {Redis};
    
    % Stockage
    \node[draw, rectangle, minimum width=3.5cm, minimum height=2.5cm] (storage) at (0,-3.5) {};
    \node at (0,-2.4) {\textbf{Stockage}};
    \node[draw, rectangle, fill=green!20, font=\scriptsize] at (0,-3.2) {Statiques};
    \node[draw, rectangle, fill=green!20, font=\scriptsize] at (0,-3.9) {Médias};
    
    % Connexions
    \draw[->, thick] (web) -- (app);
    \draw[->, thick] (app) -- (db);
    \draw[->, thick] (app) -- (storage);
\end{tikzpicture}
\caption{Diagramme de déploiement}
\end{figure}

\section{Diagramme de classes}

% Restructuration complète du diagramme de classes pour format A4 en paysage
\begin{figure}[H]
\centering
\begin{tikzpicture}[scale=0.48, transform shape, font=\tiny]

% Ligne 1
\begin{class}[text width=4cm]{User}{0,0}
    \attribute{+ id : Integer}
    \attribute{+ username : String}
    \attribute{+ email : String}
    \attribute{+ password : String}
    \attribute{+ first\_name : String}
    \attribute{+ last\_name : String}
    \attribute{+ is\_staff : Boolean}
    \operation{+ save() : void}
    \operation{+ delete() : void}
\end{class}

\begin{class}[text width=4cm]{ProfilClient}{6,0}
    \attribute{+ user : User}
    \attribute{+ date\_naissance : Date}
    \attribute{+ ville\_naissance : String}
    \attribute{+ telephone : String}
    \attribute{+ abonnement : String}
    \attribute{+ prochain\_abonnement : String}
    \attribute{+ prochaine\_facturation : Date}
    \operation{+ calculer\_decouvert() : Decimal}
\end{class}

\begin{class}[text width=4cm]{Compte}{12,0}
    \attribute{+ id : Integer}
    \attribute{+ user : User}
    \attribute{+ type\_compte : String}
    \attribute{+ solde : Decimal}
    \attribute{+ numero\_compte : String}
    \attribute{+ date\_creation : DateTime}
    \attribute{+ est\_actif : Boolean}
    \operation{+ crediter(montant) : void}
    \operation{+ debiter(montant) : void}
\end{class}

\begin{class}[text width=4cm]{Carte}{18,0}
    \attribute{+ id : Integer}
    \attribute{+ compte : Compte}
    \attribute{+ numero\_visible : String}
    \attribute{+ date\_expiration : Date}
    \attribute{+ plafond\_paiement : Integer}
    \attribute{+ plafond\_retrait : Integer}
    \attribute{+ est\_bloquee : Boolean}
    \attribute{+ sans\_contact\_actif : Boolean}
    \operation{+ bloquer() : void}
    \operation{+ debloquer() : void}
\end{class}

% Ligne 2
\begin{class}[text width=4cm]{Transaction}{0,-8}
    \attribute{+ id : Integer}
    \attribute{+ compte : Compte}
    \attribute{+ montant : Decimal}
    \attribute{+ libelle : String}
    \attribute{+ date\_execution : DateTime}
    \attribute{+ type : String}
    \attribute{+ categorie : String}
    \operation{+ executer() : void}
\end{class}

\begin{class}[text width=4cm]{Beneficiaire}{6,-8}
    \attribute{+ id : Integer}
    \attribute{+ user : User}
    \attribute{+ nom : String}
    \attribute{+ iban : String}
    \attribute{+ date\_ajout : DateTime}
    \operation{+ valider\_iban() : Boolean}
\end{class}

\begin{class}[text width=4cm]{DemandeCredit}{12,-8}
    \attribute{+ id : Integer}
    \attribute{+ user : User}
    \attribute{+ produit : String}
    \attribute{+ montant\_souhaite : Integer}
    \attribute{+ duree\_annees : Integer}
    \attribute{+ score\_calcule : Integer}
    \attribute{+ statut : String}
    \attribute{+ ia\_decision : String}
    \operation{+ calculer\_score() : Integer}
\end{class}

\begin{class}[text width=4cm]{Notification}{18,-8}
    \attribute{+ id : Integer}
    \attribute{+ user : User}
    \attribute{+ titre : String}
    \attribute{+ contenu : Text}
    \attribute{+ type : String}
    \attribute{+ est\_lu : Boolean}
    \attribute{+ date\_creation : DateTime}
    \operation{+ marquer\_lu() : void}
\end{class}

% Relations simplifiées
\draw[->] (User) -- (ProfilClient);
\draw[->] (User) -- (Compte);
\draw[->] (Compte) -- (Carte);
\draw[->] (Compte) -- (Transaction);
\draw[->] (User) to[out=270,in=90] (Beneficiaire);
\draw[->] (User) to[out=270,in=90] (DemandeCredit);
\draw[->] (User) to[out=270,in=90] (Notification);

\end{tikzpicture}
\caption{Diagramme de classes UML (vue simplifiée)}
\end{figure}

\section{Diagramme de séquence - Virement}

% Réduction et simplification du diagramme de séquence virement
\begin{figure}[H]
\centering
\begin{tikzpicture}[>=stealth, node distance=1.8cm, font=\small]
    % Acteurs et objets
    \node (client) {Client};
    \node[right=of client] (interface) {Interface};
    \node[right=of interface] (view) {View};
    \node[right=of view] (compte) {Compte};
    \node[right=of compte] (notif) {Notif};
    
    % Lignes de vie
    \draw[dashed] (client) -- ++(0,-7);
    \draw[dashed] (interface) -- ++(0,-7);
    \draw[dashed] (view) -- ++(0,-7);
    \draw[dashed] (compte) -- ++(0,-7);
    \draw[dashed] (notif) -- ++(0,-7);
    
    % Messages
    \draw[->, thick] (0,-1) -- node[above, font=\scriptsize] {1: Saisir} ++(1.8,0);
    \draw[->, thick] (1.8,-1.5) -- node[above, font=\scriptsize] {2: Valider} ++(1.8,0);
    \draw[->, thick] (3.6,-2) -- node[above, font=\scriptsize] {3: Vérifier} ++(1.8,0);
    \draw[->, thick] (5.4,-2.5) -- node[right, font=\scriptsize] {4: OK} ++(-1.8,0);
    \draw[->, thick] (3.6,-3) -- node[above, font=\scriptsize] {5: Débiter} ++(1.8,0);
    \draw[->, thick] (5.4,-3.5) -- node[above, font=\scriptsize] {6: Transaction} ++(1.8,0);
    \draw[->, thick] (3.6,-4) -- node[above, font=\scriptsize, sloped] {7: Notifier} ++(3.6,0);
    \draw[->, thick] (0,-5) -- node[above, font=\scriptsize] {8: Confirmer} ++(1.8,0);
\end{tikzpicture}
\caption{Diagramme de séquence - Virement}
\end{figure}

\section{Diagramme de séquence - Validation de crédit}

% Réduction et simplification du diagramme de séquence crédit
\begin{figure}[H]
\centering
\begin{tikzpicture}[>=stealth, node distance=2cm, font=\small]
    % Acteurs et objets
    \node (client) {Client};
    \node[right=2cm of client] (interface) {Interface};
    \node[right=2cm of interface] (view) {View};
    \node[right=2cm of view] (scoring) {Scoring};
    \node[right=2cm of scoring] (admin) {Admin};
    
    % Lignes de vie
    \draw[dashed] (client) -- ++(0,-8);
    \draw[dashed] (interface) -- ++(0,-8);
    \draw[dashed] (view) -- ++(0,-8);
    \draw[dashed] (scoring) -- ++(0,-8);
    \draw[dashed] (admin) -- ++(0,-8);
    
    % Messages
    \draw[->, thick] (0,-1) -- node[above, font=\scriptsize] {1: Formulaire} ++(2,0);
    \draw[->, thick] (2,-1.5) -- node[above, font=\scriptsize] {2: Soumettre} ++(2,0);
    \draw[->, thick] (4,-2) -- node[above, font=\scriptsize] {3: Calculer} ++(2,0);
    \draw[->, thick] (6,-2.5) -- node[right, font=\scriptsize] {4: Score} ++(-2,0);
    \draw[->, thick] (4,-3) -- node[above, font=\scriptsize] {5: EN\_ATTENTE} ++(4,0);
    \draw[->, thick] (8,-3.5) -- node[above, font=\scriptsize] {6: Examiner} ++(-4,0);
    \draw[->, thick] (4,-4) -- node[above, font=\scriptsize] {7: Décision} ++(-2,0);
    \draw[->, thick] (2,-4.5) -- node[above, font=\scriptsize] {8: Notifier} ++(-2,0);
\end{tikzpicture}
\caption{Diagramme de séquence - Validation de crédit}
\end{figure}

\section{Conclusion}
Ce chapitre a présenté l'architecture logique et physique du système, ainsi que les diagrammes UML de conception. Le prochain chapitre détaillera la base de données et les tables.

% ============================================================================
% CHAPITRE 4 : BASE DE DONNÉES
% ============================================================================
\chapter{Base de données}

\section{Introduction}
Ce chapitre décrit la structure de la base de données, les tables et leurs relations.

\section{Modèle relationnel}

\begin{figure}[H]
\centering
\begin{tikzpicture}[node distance=2cm]
    % Table User
    \node[draw, rectangle] (user) at (0,0) {
        \begin{tabular}{l}
            \textbf{auth\_user} \\
            \hline
            PK: id \\
            username \\
            email \\
            password \\
            first\_name \\
            last\_name \\
            is\_staff
        \end{tabular}
    };
    
    % Table ProfilClient
    \node[draw, rectangle] (profil) at (0,-4) {
        \begin{tabular}{l}
            \textbf{profil\_client} \\
            \hline
            PK: id \\
            FK: user\_id \\
            date\_naissance \\
            ville\_naissance \\
            telephone \\
            abonnement
        \end{tabular}
    };
    
    % Table Compte
    \node[draw, rectangle] (compte) at (6,0) {
        \begin{tabular}{l}
            \textbf{compte} \\
            \hline
            PK: id \\
            FK: user\_id \\
            type\_compte \\
            solde \\
            numero\_compte \\
            est\_actif
        \end{tabular}
    };
    
    % Table Carte
    \node[draw, rectangle] (carte) at (6,-4) {
        \begin{tabular}{l}
            \textbf{carte} \\
            \hline
            PK: id \\
            FK: compte\_id \\
            numero\_visible \\
            date\_expiration \\
            plafond\_paiement \\
            est\_bloquee
        \end{tabular}
    };
    
    % Table Transaction
    \node[draw, rectangle] (transaction) at (12,0) {
        \begin{tabular}{l}
            \textbf{transaction} \\
            \hline
            PK: id \\
            FK: compte\_id \\
            montant \\
            libelle \\
            type \\
            categorie \\
            date\_execution
        \end{tabular}
    };
    
    % Table Notification
    \node[draw, rectangle] (notification) at (12,-4) {
        \begin{tabular}{l}
            \textbf{notification} \\
            \hline
            PK: id \\
            FK: user\_id \\
            titre \\
            contenu \\
            type \\
            est\_lu
        \end{tabular}
    };
    
    % Relations
    \draw[->, thick] (user) -- (profil);
    \draw[->, thick] (user) -- (compte);
    \draw[->, thick] (compte) -- (carte);
    \draw[->, thick] (compte) -- (transaction);
    \draw[->, thick] (user) -- (notification);
\end{tikzpicture}
\caption{Modèle relationnel simplifié}
\end{figure}

\section{Dictionnaire des données}

\subsection{Table: auth\_user}
\begin{longtable}{|p{3.5cm}|p{2.5cm}|p{2cm}|p{6cm}|}
\hline
\rowcolor{banquiseblue!30}
\textbf{Champ} & \textbf{Type} & \textbf{Contraintes} & \textbf{Description} \\
\hline
\endfirsthead
\hline
\rowcolor{banquiseblue!30}
\textbf{Champ} & \textbf{Type} & \textbf{Contraintes} & \textbf{Description} \\
\hline
\endhead

id & Integer & PK, AUTO & Identifiant unique de l'utilisateur \\
\hline
username & String(150) & UNIQUE, NOT NULL & Nom d'utilisateur pour la connexion \\
\hline
email & String(254) & NOT NULL & Adresse email \\
\hline
password & String(128) & NOT NULL & Mot de passe hashé \\
\hline
first\_name & String(150) & NULL & Prénom de l'utilisateur \\
\hline
last\_name & String(150) & NULL & Nom de famille \\
\hline
is\_staff & Boolean & DEFAULT FALSE & Indicateur de rôle administrateur \\
\hline
is\_active & Boolean & DEFAULT TRUE & Compte actif ou non \\
\hline
date\_joined & DateTime & NOT NULL & Date d'inscription \\
\hline

\caption{Structure de la table auth\_user}
\end{longtable}

\subsection{Table: profil\_client}
\begin{longtable}{|p{3.5cm}|p{2.5cm}|p{2cm}|p{6cm}|}
\hline
\rowcolor{banquiseblue!30}
\textbf{Champ} & \textbf{Type} & \textbf{Contraintes} & \textbf{Description} \\
\hline
\endfirsthead
\hline
\rowcolor{banquiseblue!30}
\textbf{Champ} & \textbf{Type} & \textbf{Contraintes} & \textbf{Description} \\
\hline
\endhead

id & Integer & PK, AUTO & Identifiant unique du profil \\
\hline
user\_id & Integer & FK, UNIQUE & Référence vers auth\_user (relation 1:1) \\
\hline
date\_naissance & Date & NULL & Date de naissance du client \\
\hline
ville\_naissance & String(100) & NULL & Ville de naissance \\
\hline
telephone & String(15) & NULL & Numéro de téléphone \\
\hline
abonnement & String(15) & NOT NULL & Type d'abonnement (ESSENTIEL, PLUS, INFINITE) \\
\hline
prochain\_abonnement & String(15) & NULL & Abonnement prévu après résiliation \\
\hline
prochaine\_facturation & Date & NOT NULL & Date de la prochaine facturation \\
\hline

\caption{Structure de la table profil\_client}
\end{longtable}

\subsection{Table: compte}
\begin{longtable}{|p{3.5cm}|p{2.5cm}|p{2cm}|p{6cm}|}
\hline
\rowcolor{banquiseblue!30}
\textbf{Champ} & \textbf{Type} & \textbf{Contraintes} & \textbf{Description} \\
\hline
\endfirsthead
\hline
\rowcolor{banquiseblue!30}
\textbf{Champ} & \textbf{Type} & \textbf{Contraintes} & \textbf{Description} \\
\hline
\endhead

id & Integer & PK, AUTO & Identifiant unique du compte \\
\hline
user\_id & Integer & FK & Référence vers auth\_user \\
\hline
type\_compte & String(20) & NOT NULL & Type : COURANT, EPARGNE, PRO \\
\hline
solde & Decimal(12,2) & DEFAULT 0 & Solde actuel du compte \\
\hline
numero\_compte & String(30) & UNIQUE & IBAN du compte (format FR76...) \\
\hline
date\_creation & DateTime & AUTO & Date de création du compte \\
\hline
est\_actif & Boolean & DEFAULT TRUE & Compte actif ou fermé \\
\hline

\caption{Structure de la table compte}
\end{longtable}

\subsection{Table: carte}
\begin{longtable}{|p{3.5cm}|p{2.5cm}|p{2cm}|p{6cm}|}
\hline
\rowcolor{banquiseblue!30}
\textbf{Champ} & \textbf{Type} & \textbf{Contraintes} & \textbf{Description} \\
\hline
\endfirsthead
\hline
\rowcolor{banquiseblue!30}
\textbf{Champ} & \textbf{Type} & \textbf{Contraintes} & \textbf{Description} \\
\hline
\endhead

id & Integer & PK, AUTO & Identifiant unique de la carte \\
\hline
compte\_id & Integer & FK & Référence vers compte \\
\hline
numero\_visible & String(4) & NOT NULL & 4 derniers chiffres de la carte \\
\hline
date\_expiration & Date & NOT NULL & Date d'expiration de la carte \\
\hline
plafond\_paiement & Integer & DEFAULT 2000 & Plafond de paiement mensuel (€) \\
\hline
plafond\_retrait & Integer & DEFAULT 500 & Plafond de retrait mensuel (€) \\
\hline
est\_bloquee & Boolean & DEFAULT FALSE & Carte bloquée ou non \\
\hline
sans\_contact\_actif & Boolean & DEFAULT TRUE & Paiement sans contact activé \\
\hline
paiement\_etranger\_actif & Boolean & DEFAULT FALSE & Paiements à l'étranger autorisés \\
\hline

\caption{Structure de la table carte}
\end{longtable}

\subsection{Table: transaction}
\begin{longtable}{|p{3.5cm}|p{2.5cm}|p{2cm}|p{6cm}|}
\hline
\rowcolor{banquiseblue!30}
\textbf{Champ} & \textbf{Type} & \textbf{Contraintes} & \textbf{Description} \\
\hline
\endfirsthead
\hline
\rowcolor{banquiseblue!30}
\textbf{Champ} & \textbf{Type} & \textbf{Contraintes} & \textbf{Description} \\
\hline
\endhead

id & Integer & PK, AUTO & Identifiant unique de la transaction \\
\hline
compte\_id & Integer & FK & Référence vers compte \\
\hline
montant & Decimal(10,2) & NOT NULL & Montant de la transaction \\
\hline
libelle & String(100) & NULL & Description de la transaction \\
\hline
date\_execution & DateTime & DEFAULT NOW & Date et heure d'exécution \\
\hline
type & String(10) & NOT NULL & DEBIT ou CREDIT \\
\hline
categorie & String(20) & NOT NULL & Catégorie (ALIM, LOGEMENT, TRANSPORT, etc.) \\
\hline

\caption{Structure de la table transaction}
\end{longtable}

\subsection{Table: beneficiaire}
\begin{longtable}{|p{3.5cm}|p{2.5cm}|p{2cm}|p{6cm}|}
\hline
\rowcolor{banquiseblue!30}
\textbf{Champ} & \textbf{Type} & \textbf{Contraintes} & \textbf{Description} \\
\hline
\endfirsthead
\hline
\rowcolor{banquiseblue!30}
\textbf{Champ} & \textbf{Type} & \textbf{Contraintes} & \textbf{Description} \\
\hline
\endhead

id & Integer & PK, AUTO & Identifiant unique du bénéficiaire \\
\hline
user\_id & Integer & FK & Référence vers auth\_user \\
\hline
nom & String(100) & NOT NULL & Nom du bénéficiaire \\
\hline
iban & String(34) & NOT NULL & IBAN du bénéficiaire \\
\hline
date\_ajout & DateTime & AUTO & Date d'ajout du bénéficiaire \\
\hline

\caption{Structure de la table beneficiaire}
\end{longtable}

\subsection{Table: demande\_credit}
\begin{longtable}{|p{3.5cm}|p{2.5cm}|p{2cm}|p{6cm}|}
\hline
\rowcolor{banquiseblue!30}
\textbf{Champ} & \textbf{Type} & \textbf{Contraintes} & \textbf{Description} \\
\hline
\endfirsthead
\hline
\rowcolor{banquiseblue!30}
\textbf{Champ} & \textbf{Type} & \textbf{Contraintes} & \textbf{Description} \\
\hline
\endhead

id & Integer & PK, AUTO & Identifiant unique de la demande \\
\hline
user\_id & Integer & FK & Référence vers auth\_user \\
\hline
produit\_id & Integer & FK & Référence vers produit\_pret \\
\hline
montant\_souhaite & Integer & NOT NULL & Montant du crédit demandé (€) \\
\hline
duree\_annees & Integer & NOT NULL & Durée souhaitée en années \\
\hline
apport\_personnel & Integer & NOT NULL & Montant de l'apport personnel (€) \\
\hline
revenus\_mensuels & Integer & NOT NULL & Revenus mensuels nets (€) \\
\hline
loyer\_actuel & Integer & NULL & Loyer actuel (€) \\
\hline
dettes\_mensuelles & Integer & NOT NULL & Charges de dettes mensuelles (€) \\
\hline
enfants\_a\_charge & Integer & DEFAULT 0 & Nombre d'enfants à charge \\
\hline
emploi\_snapshot\_id & Integer & FK & Type d'emploi au moment de la demande \\
\hline
logement\_snapshot\_id & Integer & FK & Type de logement actuel \\
\hline
sante\_snapshot & String(10) & NOT NULL & État de santé (BON, MOYEN, FAIBLE) \\
\hline
score\_calcule & Integer & NULL & Score calculé par l'algorithme \\
\hline
taux\_calcule & Decimal(5,2) & NULL & Taux d'intérêt calculé (\%) \\
\hline
recommendation & String(50) & NULL & Recommandation IA \\
\hline
statut & String(20) & DEFAULT EN\_ATTENTE & EN\_ATTENTE, ACCEPTEE, REFUSEE \\
\hline
date\_demande & DateTime & AUTO & Date de soumission de la demande \\
\hline
ia\_decision & String(20) & NULL & Avis automatique de l'IA \\
\hline
mensualite\_calculee & Decimal(10,2) & NULL & Mensualité estimée (€) \\
\hline

\caption{Structure de la table demande\_credit}
\end{longtable}

\subsection{Table: notification}
\begin{longtable}{|p{3.5cm}|p{2.5cm}|p{2cm}|p{6cm}|}
\hline
\rowcolor{banquiseblue!30}
\textbf{Champ} & \textbf{Type} & \textbf{Contraintes} & \textbf{Description} \\
\hline
\endfirsthead
\hline
\rowcolor{banquiseblue!30}
\textbf{Champ} & \textbf{Type} & \textbf{Contraintes} & \textbf{Description} \\
\hline
\endhead

id & Integer & PK, AUTO & Identifiant unique de la notification \\
\hline
user\_id & Integer & FK & Référence vers auth\_user \\
\hline
titre & String(100) & NOT NULL & Titre de la notification \\
\hline
contenu & Text & NOT NULL & Contenu détaillé \\
\hline
type & String(20) & NOT NULL & VIREMENT, TRANSACTION, CREDIT, INFO \\
\hline
url & String(250) & NULL & Lien optionnel vers une page \\
\hline
est\_lu & Boolean & DEFAULT FALSE & Notification lue ou non \\
\hline
date\_creation & DateTime & AUTO & Date de création \\
\hline

\caption{Structure de la table notification}
\end{longtable}

\subsection{Table: message\_support}
\begin{longtable}{|p{3.5cm}|p{2.5cm}|p{2cm}|p{6cm}|}
\hline
\rowcolor{banquiseblue!30}
\textbf{Champ} & \textbf{Type} & \textbf{Contraintes} & \textbf{Description} \\
\hline
\endfirsthead
\hline
\rowcolor{banquiseblue!30}
\textbf{Champ} & \textbf{Type} & \textbf{Contraintes} & \textbf{Description} \\
\hline
\endhead

id & Integer & PK, AUTO & Identifiant unique du message \\
\hline
user\_id & Integer & FK & Référence vers auth\_user \\
\hline
contenu & Text & NOT NULL & Contenu du message \\
\hline
est\_admin & Boolean & DEFAULT FALSE & Message envoyé par admin ou client \\
\hline
date\_envoi & DateTime & AUTO & Date et heure d'envoi \\
\hline
est\_lu & Boolean & DEFAULT FALSE & Message lu ou non \\
\hline

\caption{Structure de la table message\_support}
\end{longtable}

\section{Schéma de la base de données}

\begin{figure}[H]
\centering
\begin{tikzpicture}[scale=0.65, every node/.style={scale=0.65}]
    % Légende
    \node at (-2,8) {\textbf{Schéma de la base de données Banquise}};
    
    % Relations principales
    \draw[->, very thick, blue] (0,6) -- node[above] {1:1} (3,6);
    \node at (1.5,6.5) {Un à un};
    
    \draw[->, very thick, green] (0,5) -- node[above] {1:N} (3,5);
    \node at (1.5,5.5) {Un à plusieurs};
    
\end{tikzpicture}
\caption{Légende du schéma de base de données}
\end{figure}

\section{Index et optimisations}

\begin{table}[H]
\centering
\begin{tabular}{|p{4cm}|p{5cm}|p{5cm}|}
\hline
\rowcolor{banquiseblue!30}
\textbf{Table} & \textbf{Index} & \textbf{Justification} \\
\hline
compte & user\_id, numero\_compte & Recherches fréquentes par utilisateur et IBAN \\
\hline
transaction & compte\_id, date\_execution & Requêtes de relevé paginées par date \\
\hline
carte & compte\_id & Affichage des cartes d'un compte \\
\hline
notification & user\_id, est\_lu & Comptage des non lues, affichage par utilisateur \\
\hline
demande\_credit & user\_id, statut & Filtrage des demandes en attente \\
\hline
beneficiaire & user\_id & Listage des bénéficiaires d'un utilisateur \\
\hline
message\_support & user\_id, date\_envoi & Tri chronologique des conversations \\
\hline
\end{tabular}
\caption{Index de la base de données}
\end{table}

\section{Conclusion}
Ce chapitre a détaillé la structure complète de la base de données avec toutes les tables, champs, contraintes et relations. Le prochain chapitre présentera l'environnement technique et les technologies utilisées.

% ============================================================================
% CHAPITRE 5 : ENVIRONNEMENT TECHNIQUE
% ============================================================================
\chapter{Environnement technique}

\section{Introduction}
Ce chapitre présente les technologies, outils et frameworks utilisés pour développer Banquise.

\section{Stack technique}

\subsection{Backend}
\begin{table}[H]
\centering
\begin{tabular}{|p{4cm}|p{4cm}|p{6cm}|}
\hline
\rowcolor{banquiseblue!30}
\textbf{Technologie} & \textbf{Version} & \textbf{Utilisation} \\
\hline
Python & 3.9+ & Langage de programmation principal \\
\hline
Django & 4.2.25 & Framework web MVC \\
\hline
django-crispy-forms & Latest & Rendu des formulaires \\
\hline
crispy-bootstrap5 & Latest & Templates Bootstrap 5 pour forms \\
\hline
django-mathfilters & Latest & Filtres mathématiques dans templates \\
\hline
\end{tabular}
\caption{Technologies Backend}
\end{table}

\subsection{Frontend}
\begin{table}[H]
\centering
\begin{tabular}{|p{4cm}|p{4cm}|p{6cm}|}
\hline
\rowcolor{banquiseblue!30}
\textbf{Technologie} & \textbf{Version} & \textbf{Utilisation} \\
\hline
HTML5 & - & Structure des pages \\
\hline
Tailwind CSS & CDN & Framework CSS utilitaire \\
\hline
Bootstrap Icons & CDN & Bibliothèque d'icônes \\
\hline
JavaScript & ES6+ & Interactivité côté client \\
\hline
Chart.js & 4.x & Graphiques et statistiques \\
\hline
\end{tabular}
\caption{Technologies Frontend}
\end{table}

\subsection{Base de données}
\begin{table}[H]
\centering
\begin{tabular}{|p{4cm}|p{4cm}|p{6cm}|}
\hline
\rowcolor{banquiseblue!30}
\textbf{Technologie} & \textbf{Version} & \textbf{Utilisation} \\
\hline
SQLite & 3.x & Base de données de développement \\
\hline
PostgreSQL & 13+ & Base de données de production (recommandé) \\
\hline
\end{tabular}
\caption{Technologies Base de données}
\end{table}

\subsection{Outils de développement}
\begin{table}[H]
\centering
\begin{tabular}{|p{4cm}|p{4cm}|p{6cm}|}
\hline
\rowcolor{banquiseblue!30}
\textbf{Outil} & \textbf{Version} & \textbf{Utilisation} \\
\hline
Git & Latest & Gestion de version \\
\hline
VS Code & Latest & Éditeur de code \\
\hline
Django Debug Toolbar & Latest & Débogage et profilage \\
\hline
pip & Latest & Gestionnaire de paquets Python \\
\hline
virtualenv / venv & Latest & Environnements virtuels Python \\
\hline
\end{tabular}
\caption{Outils de développement}
\end{table}

\newpage

\section{Architecture des fichiers}

\begin{verbatim}
Banquise2.0/
|-- Banquise/              # Configuration Django
|   |-- __init__.py
|   |-- settings.py        # Configuration principale
|   |-- urls.py            # Routes principales
|   |-- asgi.py
|   `-- wsgi.py
|-- scoring/               # Application principale
|   |-- models.py          # Modèles de données
|   |-- views.py           # Contrôleurs et logique métier
|   |-- forms.py           # Formulaires Django
|   |-- urls.py            # Routes de l'application
|   |-- utils.py           # Fonctions utilitaires
|   |-- admin.py           # Configuration Django Admin
|   `-- middleware.py      # Middleware personnalisé
|-- templates/             # Templates HTML
|   |-- base.html          # Template de base
|   |-- registration/      # Templates authentification
|   `-- scoring/           # Templates de l'app
|-- static/                # Fichiers statiques
|   |-- css/
|   |-- js/
|   `-- images/
|-- media/                 # Fichiers uploadés
|-- manage.py              # Commandes Django
`-- requirements.txt       # Dépendances Python
\end{verbatim}

\section{Configuration de sécurité}

\subsection{Paramètres Django (settings.py)}
\begin{itemize}[leftmargin=1.5cm]
    \item \texttt{DEBUG = False} en production
    \item \texttt{SECRET\_KEY} rotatée et sécurisée
    \item \texttt{ALLOWED\_HOSTS} configuré
    \item \texttt{CSRF\_COOKIE\_SECURE = True}
    \item \texttt{SESSION\_COOKIE\_SECURE = True}
    \item \texttt{SECURE\_SSL\_REDIRECT = True}
    \item \texttt{SECURE\_HSTS\_SECONDS = 31536000}
    \item \texttt{X\_FRAME\_OPTIONS = 'DENY'}
\end{itemize}

\subsection{Middleware de sécurité}
\begin{itemize}[leftmargin=1.5cm]
    \item \texttt{SecurityMiddleware} : Headers de sécurité
    \item \texttt{CsrfViewMiddleware} : Protection CSRF
    \item \texttt{AuthenticationMiddleware} : Gestion de session
    \item \texttt{MessageMiddleware} : Messages flash
\end{itemize}

\section{Déploiement}

\subsection{Configuration de production}
\begin{table}[H]
\centering
\begin{tabular}{|p{5cm}|p{9cm}|}
\hline
\rowcolor{banquiseblue!30}
\textbf{Composant} & \textbf{Configuration recommandée} \\
\hline
Serveur Web & Nginx avec SSL/TLS (Let's Encrypt) \\
\hline
Serveur Application & Gunicorn avec 4-8 workers \\
\hline
Base de données & PostgreSQL 13+ avec connexions poolées \\
\hline
Cache & Redis pour sessions et cache \\
\hline
Stockage statique & CDN (Cloudflare, AWS S3) \\
\hline
Monitoring & Sentry pour erreurs, Prometheus pour métriques \\
\hline
Logs & Centralisation avec ELK Stack ou équivalent \\
\hline
Backups & Sauvegardes quotidiennes automatisées \\
\hline
\end{tabular}
\caption{Configuration de production}
\end{table}

\subsection{Commandes de déploiement}
\begin{verbatim}
# Installer les dépendances
pip install -r requirements.txt

# Appliquer les migrations
python manage.py migrate

# Collecter les fichiers statiques
python manage.py collectstatic --noinput

# Créer un superutilisateur
python manage.py createsuperuser

# Lancer le serveur (développement)
python manage.py runserver

# Lancer avec Gunicorn (production)
gunicorn Banquise.wsgi:application --bind 0.0.0.0:8000
\end{verbatim}

\section{Tests}

\subsection{Stratégie de tests}
\begin{itemize}[leftmargin=1.5cm]
    \item \textbf{Tests unitaires} : Validation des modèles et fonctions
    \item \textbf{Tests d'intégration} : Validation des flux métier complets
    \item \textbf{Tests de sécurité} : Validation CSRF, XSS, injection SQL
    \item \textbf{Tests de performance} : Charge et temps de réponse
\end{itemize}

\subsection{Commandes de tests}
\begin{verbatim}
# Lancer tous les tests
python manage.py test

# Tests avec couverture
coverage run --source='.' manage.py test
coverage report
\end{verbatim}

\section{Environnement matériel}

\subsection{Configuration minimale (développement)}
\begin{table}[H]
\centering
\begin{tabular}{|p{5cm}|p{9cm}|}
\hline
\rowcolor{banquiseblue!30}
\textbf{Composant} & \textbf{Spécification} \\
\hline
Processeur & Intel Core i5 ou équivalent \\
\hline
RAM & 8 GB minimum \\
\hline
Disque dur & 50 GB disponibles \\
\hline
Système d'exploitation & Windows 10+, macOS 10.15+, Linux (Ubuntu 20.04+) \\
\hline
\end{tabular}
\caption{Configuration minimale}
\end{table}

\subsection{Configuration recommandée (production)}
\begin{table}[H]
\centering
\begin{tabular}{|p{5cm}|p{9cm}|}
\hline
\rowcolor{banquiseblue!30}
\textbf{Composant} & \textbf{Spécification} \\
\hline
Serveur Application & 4 vCPU, 8 GB RAM \\
\hline
Serveur Base de données & 2 vCPU, 4 GB RAM, SSD 100 GB \\
\hline
Bande passante & 100 Mbps minimum \\
\hline
Disponibilité & Load balancing, failover automatique \\
\hline
\end{tabular}
\caption{Configuration recommandée}
\end{table}

\section{Conclusion}
Ce chapitre a détaillé l'environnement technique complet, les technologies utilisées, la configuration de sécurité et les spécifications de déploiement.

% ============================================================================
% CHAPITRE 6 : CONCLUSION ET PERSPECTIVES
% ============================================================================
\chapter{Conclusion et perspectives}

\section{Synthèse du projet}
Banquise représente une solution complète de néobanque digitale, intégrant les fonctionnalités essentielles d'une banque moderne :
\begin{itemize}[leftmargin=1.5cm]
    \item Gestion multi-comptes et multi-cartes
    \item Système de virements sécurisés
    \item Scoring automatique de crédit
    \item Notifications en temps réel
    \item Support client intégré
    \item Back-office administrateur complet
\end{itemize}

Le projet a été développé en respectant les meilleures pratiques de sécurité, performance et maintenabilité, tout en adoptant une méthodologie agile permettant une évolution continue.

\section{Objectifs atteints}
\begin{itemize}[leftmargin=1.5cm]
    \item \checkmark{} Plateforme web fonctionnelle et sécurisée
    \item \checkmark{} Interface utilisateur intuitive et responsive
    \item \checkmark{} Architecture scalable et maintenable
    \item \checkmark{} Système de notifications complet
    \item \checkmark{} Scoring crédit avec IA
    \item \checkmark{} Gestion des découverts automatisée
    \item \checkmark{} Support client bidirectionnel
    \item \checkmark{} Dashboard administrateur avec statistiques
\end{itemize}

\section{Perspectives d'évolution}

\subsection{Court terme (3-6 mois)}
\begin{itemize}[leftmargin=1.5cm]
    \item Mise en place de notifications email/SMS
    \item Intégration d'un système de 2FA pour tous les utilisateurs
    \item Export PDF avancé des relevés
    \item Amélioration du module de statistiques
    \item Tests automatisés complets
\end{itemize}

\subsection{Moyen terme (6-12 mois)}
\begin{itemize}[leftmargin=1.5cm]
    \item Application mobile native (iOS/Android)
    \item KYC (Know Your Customer) automatisé
    \item Intégration de paiements tiers (PayPal, Stripe)
    \item Système de cashback et récompenses
    \item Comptes joints et multi-utilisateurs
    \item Module d'épargne automatique
\end{itemize}

\subsection{Long terme (1-2 ans)}
\begin{itemize}[leftmargin=1.5cm]
    \item Agrégation de comptes externes
    \item Assistant IA pour conseil financier
    \item Marketplace de produits financiers partenaires
    \item Investissements (actions, crypto, obligations)
    \item Open Banking (PSD2) et API publique
    \item Expansion internationale
\end{itemize}

\section{Risques et mitigation}

\begin{table}[H]
\centering
\begin{tabular}{|p{4cm}|p{5cm}|p{5cm}|}
\hline
\rowcolor{banquiseblue!30}
\textbf{Risque} & \textbf{Impact} & \textbf{Mitigation} \\
\hline
Faille de sécurité & Très élevé & Audits réguliers, pen testing, bug bounty \\
\hline
Surcharge serveur & Élevé & Scalabilité horizontale, CDN, cache \\
\hline
Perte de données & Très élevé & Backups quotidiens, réplication \\
\hline
Non-conformité RGPD & Élevé & Audit légal, DPO dédié \\
\hline
Fraude utilisateur & Moyen & Monitoring IA, alertes anomalies \\
\hline
\end{tabular}
\caption{Analyse des risques}
\end{table}

\section{Conclusion finale}
Banquise constitue une base solide pour une néobanque moderne et évolutive. Le projet démontre la faisabilité technique d'une plateforme bancaire complète en mode web, tout en laissant de nombreuses opportunités d'amélioration et d'innovation.

L'architecture modulaire et la méthodologie agile adoptées permettront d'intégrer facilement de nouvelles fonctionnalités et de s'adapter aux évolutions du marché et aux besoins des utilisateurs.

% ============================================================================
% ANNEXES
% ============================================================================
\appendix

\chapter{Glossaire technique}

\begin{description}[style=nextline, leftmargin=4cm]
    \item[API] Application Programming Interface : interface de programmation
    \item[CSRF] Cross-Site Request Forgery : attaque par requête falsifiée
    \item[CRUD] Create, Read, Update, Delete : opérations de base
    \item[DTO] Data Transfer Object : objet de transfert de données
    \item[HTTPS] HTTP Secure : protocole de communication sécurisé
    \item[JWT] JSON Web Token : jeton d'authentification
    \item[MVC] Model-View-Controller : patron d'architecture
    \item[ORM] Object-Relational Mapping : mapping objet-relationnel
    \item[REST] Representational State Transfer : style d'architecture API
    \item[SQL] Structured Query Language : langage de requête
    \item[TLS] Transport Layer Security : protocole de sécurité
    \item[WSGI] Web Server Gateway Interface : interface serveur web Python
    \item[XSS] Cross-Site Scripting : injection de script malveillant
\end{description}

\chapter{Bibliographie et références}

\begin{enumerate}
    \item Django Documentation - \url{https://docs.djangoproject.com/}
    \item Tailwind CSS Documentation - \url{https://tailwindcss.com/docs}
    \item RGPD - Règlement Général sur la Protection des Données
    \item OWASP Top 10 - Risques de sécurité web
    \item PSD2 - Payment Services Directive 2
    \item PostgreSQL Documentation - \url{https://www.postgresql.org/docs/}
    \item Gunicorn Documentation - \url{https://docs.gunicorn.org/}
\end{enumerate}

% Ajout de sections vides pour les futures captures d'écran
\chapter{Captures d'écran de l'interface}

\section{Interface utilisateur}

\subsection{Page d'accueil et connexion}
\begin{figure}[H]
    \centering
    \includegraphics[width=\textwidth]{accueil_connexion.png}
    \caption{Page d'accueil et écran de connexion}
    \label{fig:accueil-connexion}
\end{figure}

\subsection{Dashboard client}
\begin{figure}[H]
    \centering
    \includegraphics[width=\textwidth]{dashboard_client.png}
    \caption{Dashboard client}
    \label{fig:dashboard-client}
\end{figure}

\subsection{Gestion des cartes}
\begin{figure}[H]
    \centering
    \includegraphics[width=\textwidth]{gestion_cartes.png}
    \caption{Interface de gestion des cartes}
    \label{fig:gestion-cartes}
\end{figure}

\subsection{Formulaire de virement}
\begin{figure}[H]
    \centering
    \includegraphics[width=\textwidth]{formulaire_virement.png}
    \caption{Formulaire de virement}
    \label{fig:formulaire-virement}
\end{figure}

\subsection{Simulation de crédit}
\begin{figure}[H]
    \centering
    \includegraphics[width=\textwidth]{simulation_credit.png}
    \caption{Simulation de crédit}
    \label{fig:simulation-credit}
\end{figure}

\subsection{Centre de notifications}
\begin{figure}[H]
    \centering
    \includegraphics[width=\textwidth]{centre_notifications.png}
    \caption{Centre de notifications}
    \label{fig:centre-notifications}
\end{figure}

\section{Interface administrateur}

\subsection{Dashboard admin avec statistiques}
\begin{figure}[H]
    \centering
    \includegraphics[width=\textwidth]{admin_dashboard_statistiques.png}
    \caption{Dashboard administrateur avec statistiques}
    \label{fig:admin-dashboard}
\end{figure}

\subsection{Console de validation des crédits}
\begin{figure}[H]
    \centering
    \includegraphics[width=\textwidth]{admin_console_validation_credits.png}
    \caption{Console de validation des crédits}
    \label{fig:admin-console-credits}
\end{figure}

\subsection{Gestion des comptes et cartes}
\begin{figure}[H]
    \centering
    \includegraphics[width=\textwidth]{admin_gestion_comptes_cartes.png}
    \caption{Gestion des comptes et des cartes (administration)}
    \label{fig:admin-gestion-comptes-cartes}
\end{figure}

\subsection{Interface de support client}
\begin{figure}[H]
    \centering
    \includegraphics[width=\textwidth]{admin_support_client.png}
    \caption{Interface de support client (côté admin)}
    \label{fig:admin-support-client}
\end{figure}

\end{document}
